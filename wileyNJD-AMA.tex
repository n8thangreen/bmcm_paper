\documentclass[AMA,STIX1COL]{WileyNJD-v2}

\articletype{Article Type}%

\received{26 April 2016}
\revised{6 June 2016}
\accepted{6 June 2016}

\raggedbottom

\begin{document}

\title{Bayesian hierarchical mixture cure modelling \protect\thanks{This is an example for title footnote.}}

\author[1]{Nathan Green*}

\author[2,3]{Gianluca Baio}

\author[3]{Author Three}

\authormark{N Green \textsc{et al}}

\address[1]{\orgdiv{Department of Statistical Science}, \orgname{UCL}, \orgaddress{\state{London}, \country{UK}}}

\address[2]{\orgdiv{Org Division}, \orgname{Org Name}, \orgaddress{\state{State name}, \country{Country name}}}

\address[3]{\orgdiv{Org Division}, \orgname{Org Name}, \orgaddress{\state{State name}, \country{Country name}}}

\corres{*Nathan Green \email{n.green@ucl.ac.uk}}

\presentaddress{}

\abstract[Summary]{}

\keywords{Bayesian, survival analysis}

\jnlcitation{\cname{%
\author{Green N.}, 
\author{G. Baio}, and 
\author{T. Woollings}} (\cyear{2021}), 
\ctitle{}, \cvol{2017;00:1--6}.}

\maketitle

\footnotetext{\textbf{Abbreviations:} MCM, mixture cure model; APC, antigen-presenting cells}


\section{Introduction}\label{sec1}

set the scene
more prevalent in HEA
plateau in events

BMC to give background

Q. can we do better?
i) PFS from OS
ii) pooled fraction

Why are MCM important, specifically in health economics?
Brief review of existing applications
Our innovative proposal --- joint model of PFS and OS, which has the double advantage of borrowing information (eg the likely more mature PFS data to inform the highly censored OS) *and* obtaining a pooled cure rate.


\section{Motivating example}\label{sec5}

long term
different cut points

Description of the BMS data
How it compares with dataset prevalent in our field/applications


\section{Methods}\label{sec5}

implications

Description of "standard" model and the frequentist approach typically used
Description of our proposal to model jointly PFS and OS - data model/priors/inference


\section{Application}\label{sec5}

Back to BMS data and results, with various scenarios/distributions and all the plots and tables, similar to the ones Nathan has already provided


\section{Discussion}\label{sec5}

Lorem ipsum dolor sit amet

%\backmatter

\section*{Acknowledgments}
This is acknowledgment text.\cite{Kenamond2013} Provide text here. This is acknowledgment text. Provide text here. This 
\subsection*{Author contributions}

This is an author contribution text.

\subsection*{Financial disclosure}

None reported.

\subsection*{Conflict of interest}

The authors declare no potential conflict of interests.


\section*{Supporting information}

The following supporting information is available as part of the online article:

\noindent
\textbf{Figure S1.}
{500{\uns}hPa geopotential anomalies for GC2C calculated against the ERA Interim reanalysis. The period is 1989--2008.}


\appendix

\section{Section title of first appendix\label{app1}}

with normal text font. Refer below example:

\begin{lstlisting}[caption={Descriptive Caption Text},label=DescriptiveLabel]
for i:=maxint to 0 do
\end{lstlisting}

%\nocite{*}% Show all bib entries - both cited and uncited; comment this line to view only cited bib entries;
\bibliography{wileyNJD-AMA}%

\clearpage

\section*{Author Biography}

\begin{biography}{\includegraphics[width=66pt,height=86pt,draft]{empty}}{\textbf{Author Name.} This is sample author is sample author biography text.}
\end{biography}

\end{document}
